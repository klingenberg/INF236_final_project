\documentclass[UKenglish]{beamer}


\usetheme{UiB}

\usepackage{algorithm}
\usepackage[noend]{algpseudocode}
\usepackage[utf8]{inputenx} % For æ, ø, å
\usepackage{csquotes}       % Quotation marks
\usepackage{microtype}      % Improved typography
\usepackage{amssymb}        % Mathematical symbols
\usepackage{mathtools}      % Mathematical symbols
\usepackage[absolute, overlay]{textpos} % Arbitrary placement
\setlength{\TPHorizModule}{\paperwidth} % Textpos units
\setlength{\TPVertModule}{\paperheight} % Textpos units
\usepackage{tikz}
\usetikzlibrary{overlay-beamer-styles}  % Overlay effects for TikZ
\usepackage{geometry}
\usepackage{xcolor}
\usepackage{pgfplots}
\usepackage{amsmath}
\usepackage{amsfonts}
\usepackage[backend=biber]{biblatex}
\usepackage{caption}
\usepackage{subcaption}


\author{Kate\v{r}ina \v{C}\'{i}\v{z}kov\'{a}, Luca Klingenberg}
\title{Parallel Matrix Multiplication}
\subtitle{Final Project\\ INF236: Parallel Programming}


\definecolor{gradient1}{HTML}{833ab4}
\definecolor{gradient2}{HTML}{fd1d1d}
\definecolor{gradient3}{HTML}{fc8a3b}
\definecolor{gradient4}{HTML}{fcb045}
\definecolor{gradient0}{HTML}{005ab3}

\renewcommand{\algorithmicrequire}{\textbf{Input:}}
\renewcommand{\algorithmicensure}{\textbf{Output:}}


\begin{document}


\section{Introduction}
% Use
%
%     \begin{frame}[allowframebreaks]{Title}
%
% if the TOC does not fit one frame.
\begin{frame}{Table of contents}
    \tableofcontents[currentsection]
\end{frame}

\begin{frame}{Introduction}
	\begin{itemize}
        \item
        Complexity: $\mathcal{O}(n^3)$
        \item 
        Consecutive memory access
    \end{itemize}
    
	\begin{algorithm}[H] 
\caption{matrix multiplication}
\label{alg:matmul}
\begin{algorithmic}[1]
\Require{$\mathbf{A}, \mathbf{B}$} %Input
\Ensure{$\mathbf{C}$ (the resulting matrix)} %Output
\Statex
\Function{matmul}{$\mathbf{A}, \mathbf{B}$}
	\For{$i=0, \ldots, n-1$}
		\For {$j=0, \ldots, n-1$}
			\State {$c[i][j] = 0$}
		\EndFor
		\For{$k=0, \ldots, n-1$}
			\For{$j=0, \ldots, n-1$}
				\State {$c[i][j] += a[i][k] \cdot b[k][j]$}
			\EndFor
		\EndFor
	\EndFor
	\State \Return {$\mathbf{C}$}
\EndFunction
\end{algorithmic}
\end{algorithm}
\end{frame}

\section{Strassen's Algorithm}

\begin{frame}{Strassen's Algorithm}

\begin{itemize}
        \item
        Blockwise matrix multiplication
        \begin{align*}
\mathbf{A} \cdot \mathbf{B} &=
\begin{pmatrix}
\mathbf{A}_{00} & \mathbf{A}_{01} \\
\mathbf{A}_{10} & \mathbf{A}_{11} 
\end{pmatrix}
\cdot
\begin{pmatrix}
\mathbf{B}_{00} & \mathbf{B}_{01} \\
\mathbf{B}_{10} & \mathbf{B}_{11} 
\end{pmatrix} \\
&=
\begin{pmatrix}
\mathbf{A}_{00}\mathbf{B}_{00}+\mathbf{A}_{01}\mathbf{B}_{10} & \mathbf{A}_{00}\mathbf{B}_{01}+\mathbf{A}_{01}\mathbf{B}_{11} \\
\mathbf{A}_{10}\mathbf{B}_{00}+\mathbf{A}_{11}\mathbf{B}_{10} & \mathbf{A}_{10}\mathbf{B}_{01}+\mathbf{A}_{11}\mathbf{B}_{11} 
\end{pmatrix} \\
&=
\begin{pmatrix}
\mathbf{C}_{00} & \mathbf{C}_{01} \\
\mathbf{C}_{10} & \mathbf{C}_{11} 
\end{pmatrix}
= \mathbf{C}
\end{align*}
	\item 
     Idea: reduce number of multiplications from 8 to 7
     \item 
     Complexity of Strassen's algorithm: 
    $\mathcal{O}(n^{\log_2{7}}) \approx \mathcal{O}(n^{2.8073})$
    \item
    Can be further improved by reusing intermediate results of additions and subtractions (Winograd's algorithm)
    \item
  	Input matrices are zero-padded until the next power of two
    \end{itemize}
\end{frame}


\begin{frame}{Strassen's Algorithm}
\begin{center}
    \scalebox{0.75}{
    \begin{minipage}{0.7\linewidth}
\begin{algorithm}[H] 
\caption{Strassen's matrix multiplication}
\label{alg:strassen}
\begin{algorithmic}[1]
\Require{$\mathbf{A}, \mathbf{B}$} %Input
\Ensure{$\mathbf{C}$ (the resulting matrix)} %Output
\Statex
\Function{strassen}{$\mathbf{A}, \mathbf{B}, n$}
	\If {n == cutoff}
		\State \Return \Call{matmul}{$\mathbf{A}, \mathbf{B}$}
	\EndIf
	\State {$\mathbf{P}_1 = \Call{strassen}{\mathbf{A}_{00} + \mathbf{A}_{11},  \mathbf{B}_{00} + \mathbf{B}_{11}, \frac{n}{2}$}}
	\State {$\mathbf{P}_2 = \Call{strassen}{\mathbf{A}_{10} + \mathbf{A}_{11},  \mathbf{B}_{00}, \frac{n}{2}$}}
	\State {$\mathbf{P}_3 = \Call{strassen}{\mathbf{A}_{00},  \mathbf{B}_{01} - \mathbf{B}_{11}, \frac{n}{2}$}}
	\State {$\mathbf{P}_4 = \Call{strassen}{\mathbf{A}_{11},  \mathbf{B}_{10} - \mathbf{B}_{00}, \frac{n}{2}$}}
	\State {$\mathbf{P}_5 = \Call{strassen}{\mathbf{A}_{00} + \mathbf{A}_{01},  \mathbf{B}_{11}, \frac{n}{2}$}}
	\State {$\mathbf{P}_6 = \Call{strassen}{\mathbf{A}_{10} - \mathbf{A}_{00},  \mathbf{B}_{00} + \mathbf{B}_{01}, \frac{n}{2}$}}
	\State {$\mathbf{P}_7 = \Call{strassen}{\mathbf{A}_{01} - \mathbf{A}_{11},  \mathbf{B}_{10} + \mathbf{B}_{11}, \frac{n}{2}$}}
	\State {$\mathbf{C}_{00} = \mathbf{P}_1 + \mathbf{P}_4 - \mathbf{P}_5 + \mathbf{P}_7$}
	\State {$\mathbf{C}_{01} = \mathbf{P}_3 + \mathbf{P}_5$}
	\State {$\mathbf{C}_{10} = \mathbf{P}_2 + \mathbf{P}_4$}
	\State {$\mathbf{C}_{11} = \mathbf{P}_1 - \mathbf{P}_2 + \mathbf{P}_3 + \mathbf{P}_6$}
	\State \Return {$\mathbf{C}$}
\EndFunction
\end{algorithmic}
\end{algorithm}
\end{minipage}%
    }
  \end{center}
\end{frame}

\subsection{Different values for the cutoff level}


\begin{frame}{Different values for the cutoff level}
\begin{figure}[h!]
\center
\begin{tikzpicture}
\begin{axis}[
    xlabel={submatrix dimension at cuttoff},
    ylabel={speedup},
    xmin=0, xmax=512,
    ymin=0, ymax=4,
    xtick={1, 2, 4, 8, 16, 32, 64, 128, 256, 512},
    xticklabels={1, 2, 4, 8, 16, 32, 64, 128, 256},
    ytick={0, 0.5, 1, 1.5, 2, 2.5, 3, 3.5, 4},
	xmode=log,
    log basis x={2},
    legend pos=outer north east,
    xmajorgrids=true,
    ymajorgrids=true,
    grid style=dashed,
]

	% running time of normal sequential: 22.729855
	\addplot[color=gradient1, mark=*, only marks] table [x=cutoff, y=size1, col sep=comma] {cutoffs.csv};
	% running time of normal sequential: 1.397876
	\addplot[color=gradient2, mark=*, only marks] table [x=cutoff, y=size2, col sep=comma] {cutoffs.csv};
	% running time of normal sequential: 0.177942
	\addplot[color=gradient3, mark=*, only marks] table [x=cutoff, y=size3, col sep=comma] {cutoffs.csv};
	% running time of normal sequential: 0.032513
	\addplot[color=gradient4, mark=*, only marks] table [x=cutoff, y=size4, col sep=comma] {cutoffs.csv};

    \legend{2048$\times$2048, 1024$\times$1024, 512$\times$512, 256$\times$256}
\end{axis}
\end{tikzpicture}
\end{figure}
\end{frame}

\subsection{Z-ordering}

\begin{frame}{Z-ordering}
	\vspace{0.5cm}
	\begin{figure}[htbp]
\centerline{\includegraphics[scale=.16]{z_ordering.pdf}}
\centerline{\includegraphics[scale=.2775]{mem_ordering.pdf}}
\end{figure}
\end{frame}

\begin{frame}{Z-ordering}
\begin{itemize}
		\item
		Matrices at cutoff level are stored row-wise
		\vspace{0.25cm}
			\begin{figure}[htbp]
				\centerline{\includegraphics[scale=.5]{partly_z_ordering.pdf}}
			\end{figure}
		\item
		Only pointers instead of the actual submatrices are passed \\ 
		to the recursive calls
		\end{itemize}
\end{frame}

\section{Parallelization}

\subsection{Parallel matrix multiplication}

\begin{frame}{Parallelization: Matrix multiplication}
	\begin{algorithm}[H] 
	\caption{matrix multiplication}
	\label{alg:matmul}
	\begin{algorithmic}[1]
	\Require{$\mathbf{A}, \mathbf{B}$} %Input
	\Ensure{$\mathbf{C}$ (the resulting matrix)} 
	\Statex \\ %Output 
	\#pragma omp parallel for
	\Function{matmul}{$\mathbf{A}, \mathbf{B}$}
	\For{$i=0, \ldots, n-1$}
		\For {$j=0, \ldots, n-1$}
			\State {$c[i][j] = 0$}
		\EndFor
		\For{$k=0, \ldots, n-1$}
			\For{$j=0, \ldots, n-1$}
				\State {$c[i][j] += a[i][k] \cdot b[k][j]$}
			\EndFor
		\EndFor
	\EndFor
	\State \Return {$\mathbf{C}$}
	\EndFunction
	\end{algorithmic}
	\end{algorithm}
\end{frame}


\subsection{Parallel Strassen}

\begin{frame}{Parallelization: Strassen}
\begin{itemize}
	\item
	2-layers variant
	\begin{itemize}
		\item
			Use 2 hardcoded recursion levels of Strassen
		\item
			Execute all additions and subtractions in parallel
		\item 
			At the cutoff level, execute all matrix multiplications in parallel	
	\end{itemize}
	\vspace{1cm}
	\item
	Recursive variant
	\begin{itemize}
		\item
			Use several recursive calls of Strassen
		\item
			Execute all additions and subtractions in parallel
		\item 
			At the cutoff level, execute all matrix multiplications in parallel
		\item 
			Strassen is recursively applied until the submatrices have a dimension of 512$\times$512	
	\end{itemize}
\end{itemize}
\end{frame}



\section{Experiments}

\begin{frame}{Experiments}
	\vspace{.2cm}
	\centering
	\begin{tikzpicture}
\begin{axis}[
	title=Comparing Strassen algorithm variants (256$\times$256),
    xlabel={\#threads},
    ylabel={speedup},
    xmin=0, xmax=8,
    ymin=0, ymax=26,
    xtick={0, 1, 2, 3, 4, 5, 6, 7},
    xticklabels={0, 1, 2, 5, 10, 20, 40, 60},
    ytick={0, 5, 10, 15, 20, 25},
    legend pos=north west,
    xmajorgrids=true,
    ymajorgrids=true,
    grid style=dashed,
]
	
	\addplot[color=gradient1, mark=*, only marks] table [x=config, y=strassen_vs_simple_parallel, col sep=comma] {result256.csv};
	
	\addplot[color=gradient2, mark=*, only marks] table [x=config, y=strassen_vs_strassen_2layers, col sep=comma] {result256.csv};
	
	\addplot[color=gradient4, mark=*, only marks] table [x=config, y=strassen_vs_strassen_recursive, col sep=comma] {result256.csv};
    
    \legend{parallel matmul, parallel 2 layers Strassen, parallel recursive Strassen}

\end{axis}
\end{tikzpicture}
\end{frame}

\begin{frame}{Experiments}
	\vspace{.2cm}
	\centering
	\begin{tikzpicture}
\begin{axis}[
	title=Comparing Strassen algorithm variants (4096$\times$4096),
    xlabel={\#threads},
    ylabel={speedup},
    xmin=0, xmax=8,
    ymin=0, ymax=26,
    xtick={0, 1, 2, 3, 4, 5, 6, 7},
    xticklabels={0, 1, 2, 5, 10, 20, 40, 60},
    ytick={0, 5, 10, 15, 20, 25},
    legend pos=north west,
    xmajorgrids=true,
    ymajorgrids=true,
    grid style=dashed,
]
	
	\addplot[color=gradient1, mark=*, only marks] table [x=config, y=strassen_vs_simple_parallel, col sep=comma] {result4096.csv};
	
	\addplot[color=gradient2, mark=*, only marks] table [x=config, y=strassen_vs_strassen_2layers, col sep=comma] {result4096.csv};
	
	\addplot[color=gradient4, mark=*, only marks] table [x=config, y=strassen_vs_strassen_recursive, col sep=comma] {result4096.csv};
    
    \legend{parallel matmul, parallel 2 layers Strassen, parallel recursive Strassen}

\end{axis}
\end{tikzpicture}
\end{frame}

\begin{frame}{Experiments}
	\vspace{.2cm}
	\centering
	\begin{tikzpicture}
\begin{axis}[
	title=Comparing Strassen algorithm variants (8192$\times$8192),
    xlabel={\#threads},
    ylabel={speedup},
    xmin=0, xmax=8,
    ymin=0, ymax=26,
    xtick={0, 1, 2, 3, 4, 5, 6, 7},
    xticklabels={0, 1, 2, 5, 10, 20, 40, 60},
    ytick={0, 5, 10, 15, 20, 25},
    legend pos=north west,
    xmajorgrids=true,
    ymajorgrids=true,
    grid style=dashed,
]
	
	\addplot[color=gradient1, mark=*, only marks] table [x=config, y=strassen_vs_simple_parallel, col sep=comma] {result8192.csv};
	
	\addplot[color=gradient2, mark=*, only marks] table [x=config, y=strassen_vs_strassen_2layers, col sep=comma] {result8192.csv};
	
	\addplot[color=gradient4, mark=*, only marks] table [x=config, y=strassen_vs_strassen_recursive, col sep=comma] {result8192.csv};
    
    \legend{parallel matmul, parallel 2 layers Strassen, parallel recursive Strassen}

\end{axis}
\end{tikzpicture}
\end{frame}


\begin{frame}{Experiments}
	\vspace{.2cm}
    \scalebox{0.85}{
    \begin{minipage}{0.7\linewidth}
\begin{tikzpicture}
\begin{semilogyaxis}[
	title=Running time for matrices of different sizes,
    xlabel={matrix size},
    ylabel={running time in seconds},
    ymin=0, ymax=400,
    legend pos=outer north east,
    %legend style={at={(0.5,-0.2)},anchor=north},
    xmajorgrids=true,
    ymajorgrids=true,
    grid style=dashed,
]

    \addplot[color=gradient0, mark=*, only marks] table [x=size, y=simple, col sep=comma] {results_diff_size.csv};
	
	\addplot[color=gradient1, mark=*, only marks] table [x=size, y=strassen, col sep=comma] {results_diff_size.csv};
	
	\addplot[color=gradient2, mark=*, only marks] table [x=size, y=simple_parallel, col sep=comma] {results_diff_size.csv};
	
	\addplot[color=gradient3, mark=*, only marks] table [x=size, y=strassen_2layers, col sep=comma] {results_diff_size.csv};
	
	\addplot[color=gradient4, mark=*, only marks] table [x=size, y=strassen_recursive, col sep=comma] {results_diff_size.csv};
    
    \legend{matmul, Strassen, parallel matmul, parallel 2 layers Strassen, parallel recursive Strassen }
\end{semilogyaxis}
\end{tikzpicture}
\end{minipage}%
    }
\end{frame}


\section{Conclusion}

\begin{frame}{Conclusion}

\end{frame}


\if{}
\section{Mathematics}
\subsection{Theorem}


\begin{frame}{Mathematics}
    \begin{theorem}[Fermat's little theorem]
        For a prime~\(p\) and \(a \in \mathbb{Z}\) it holds that \(a^p \equiv a \pmod{p}\).
    \end{theorem}

    \begin{proof}
        The invertible elements in a field form a group under multiplication.
        In particular, the elements
        \begin{equation*}
            1, 2, \ldots, p - 1 \in \mathbb{Z}_p
        \end{equation*}
        form a group under multiplication modulo~\(p\).
        This is a group of order \(p - 1\).
        For \(a \in \mathbb{Z}_p\) and \(a \neq 0\) we thus get \(a^{p-1} = 1 \in \mathbb{Z}_p\).
        The claim follows.
    \end{proof}
\end{frame}


\subsection{Example}


\begin{frame}{Mathematics}
    \begin{example}
        The function \(\phi \colon \mathbb{R} \to \mathbb{R}\) given by \(\phi(x) = 2x\) is continuous at the point \(x = \alpha\),
        because if \(\epsilon > 0\) and \(x \in \mathbb{R}\) is such that \(\lvert x - \alpha \rvert < \delta = \frac{\epsilon}{2}\),
        then
        \begin{equation*}
            \lvert \phi(x) - \phi(\alpha)\rvert = 2\lvert x - \alpha \rvert < 2\delta = \epsilon.
        \end{equation*}
    \end{example}
\end{frame}


\section{Highlighting}
\SectionPage


\begin{frame}{Highlighting}
    Sometimes it is useful to \alert{highlight} certain words in the text.

    \begin{alertblock}{Important message}
        If a lot of text should be \alert{highlighted}, it is a good idea to put it in a box.
    \end{alertblock}

    It is easy to match the \structure{colour theme}.
\end{frame}


\section{Lists}


\begin{frame}{Lists}
    \begin{itemize}
        \item
        Bullet lists are marked with a red box.
    \end{itemize}

    \begin{enumerate}
        \item
        \label{enum:item}
        Numbered lists are marked with a white number inside a red box.
    \end{enumerate}

    \begin{description}
        \item[Description] highlights important words with red text.
    \end{description}

    Items in numbered lists like \enumref{enum:item} can be referenced with a red box.

    \begin{example}
        \begin{itemize}
            \item
            Lists change colour after the environment.
        \end{itemize}
    \end{example}
\end{frame}


\section{Effects}


\begin{frame}{Effects}
    \begin{columns}[onlytextwidth]
        \begin{column}{0.49\textwidth}
            \begin{enumerate}[<+-|alert@+>]
                \item
                Effects that control

                \item
                when text is displayed

                \item
                are specified with <> and a list of slides.
            \end{enumerate}

            \begin{theorem}<2>
                This theorem is only visible on slide number 2.
            \end{theorem}
        \end{column}
        \begin{column}{0.49\textwidth}
            Use \textbf<2->{textblock} for arbitrary placement of objects.

            \pause
            \medskip

            It creates a box
            with the specified width (here in a percentage of the slide's width)
            and upper left corner at the specified coordinate (x, y)
            (here x is a percentage of width and y a percentage of height).
        \end{column}
    \end{columns}
    
    \begin{textblock}{0.3}(0.45, 0.55)
        \includegraphics<1, 3>[width = \textwidth]{UiB-images/UiB-emblem}
    \end{textblock}
\end{frame}


\section{References}


\begin{frame}[allowframebreaks]{References}
    \begin{thebibliography}{}

        % Article is the default.
        \setbeamertemplate{bibliography item}[book]

        \bibitem{Hartshorne1977}
        Hartshorne, R.
        \newblock \emph{Algebraic Geometry}.
        \newblock Springer-Verlag, 1977.

        \setbeamertemplate{bibliography item}[article]

        \bibitem{Helso2020}
        Helsø, M.
        \newblock \enquote{Rational quartic symmetroids}.
        \newblock \emph{Adv. Geom.}, 20(1):71--89, 2020.

        \setbeamertemplate{bibliography item}[online]

        \bibitem{HR2018}
        Helsø, M.\ and Ranestad, K.
        \newblock \emph{Rational quartic spectrahedra}, 2018.
        \newblock \url{https://arxiv.org/abs/1810.11235}

        \setbeamertemplate{bibliography item}[triangle]

        \bibitem{AM1969}
        Atiyah, M.\ and Macdonald, I.
        \newblock \emph{Introduction to commutative algebra}.
        \newblock Addison-Wesley Publishing Co., Reading, Mass.-London-Don
        Mills, Ont., 1969

        \setbeamertemplate{bibliography item}[text]

        \bibitem{Artin1966}
        Artin, M.
        \newblock \enquote{On isolated rational singularities of surfaces}.
        \newblock \emph{Amer. J. Math.}, 80(1):129--136, 1966.

    \end{thebibliography}
\end{frame}

\fi{}

\end{document}